python directiry/file.py %运行文件
exit()  %退出解释器

ctrl + d %复制当前行代码
shift + alt + up/down %移动行
Ctrl + f %搜索
ctrl + / %添加解除注释
alt + enter %自动导包

#  %加单行注释
“““     ”””   % 加多行注释, 定义多行字符串
\ %转义字符,解除效应
print("hjhjhjh",end='')  %输出不换行
print(“艰苦艰苦就看见\t开机开卷考”)  %对齐
print()  %控制换行
time.sleep(500000) %暂停执行

type()  % 查看数据类型
数据类型:string 字符串
          int    整数
          float  小数
数据类型转换:str(), int(), float()

运算符:+-*/  // %取整除
              %  %取余
              ** %指数
              ~= %x = x ~ a

字符串拼接:“阿”+a 
            "  %s  %4d   %5.2f    %.3f    " %(A, B, C, D)      %占位,转值的类型
            f"内容{变量}"   %不限数据类型,不控制精度

input(“你是?”)  %交互输入


第三章
if 1 > 2:
____asa
elif 2==4:
____hjhjhjhjh
else:
____asasasa

while 1==2:
____ioioioio
flag = True

for i in jkjkjkjk:
    klklklk
range(5) = [0 1 2 3 4]
range(5,10) = [5 6 7 8 9]
range(4,10,2)=[4 6 8]

第五章 函数
def function(data1,data2):
    klklklklk
    return 返回值
if not None %None = False
global num  % 函数内宣称num为全局变量

第六章  数据容器
list  [a,b]
my_list.index(元素) %找指标
my_list.insert(2, a )  %插入
my_list.append(a)  %追加
my_list.extend(A)  %追加一批元素
del my_list[3]    % 删除元素
my_list.pop(30)   % 删除元素,此为取出的元素
my_list.remove(a)   %删除第一个出现的元素
my_list.clear()   %清空
my_list.count(a)   %统计数量
len(my_list)    %长度

tuple (a,b)  %元组

new_my_str = my_str.replace("it", "op")  % 替换获取
my_str_list = my_str.split(" ")  %按 分割
new_my_str = my_str.strip(12)  %删前后的1和2,()时,去空格

list[2:9%#不含#% :2]  %取子序列

set {a,b}  %集合 set() 为空集
my_set.add(c)  %添加元素
my_set.remove(c)  %删除元素
my_set.pop()  %随机取元素
my_set.clear()  %清空元素
my_set1.diffrence(my_set2) %A\B  不影响AB
my_set1.diffrence_update(my_set2) %A\B  影响A
my_set1.union(my_set2) %A并B

dictionary  {"a" : 1 , "b" : 2}  %字典 空字典{}
my_dict["a"] %= 1
my_dict["c"] = 4  %新增或更新元素
my_dict.pop(key)  % 取其value,并删除
my_dict.clear()
my_dict.keys()  %取全部key,用于遍历,或直接for循环

list/str/tuple/set()  %类型转换
sorted(my_list%,reverse = Ture%)  % 排序为列表,%倒序

第七章  
return 1 , 2
def info(*args) % 位置不定长。元组
def info(**kwargs)  % 关键字不定长, 字典
def phi(compute)   %泛函
function(lambda x,y : x + y )  %简单调用匿名函数

第八章
open("directory","r/w/a",encoding = "UTF-8")  %打开文件,w,a可创建
f.read(10)  %读取字节 ()读全部  接续读
f.readlines()  %读取行,装于列表  \n 换行  ,也为for的读取形式
f.readline()  %读一行
f.close()
with open() as f:  %自动删除
____jfjffjjf  
f.write("dldldldl")  %暂存区
f.flush()   %刷新保存

第九章  异常
try:
____opopopop
except:
____zizizizi
else:
____klklkl
finally:
opipjohjo
except (NameError,。。) as e:  %捕获制定异常 ,e为异常信息  Exception  为全部异常

import time  %ctrl + click to see time.py     [from time import * as t]
__main__变量  % 不运行执行代码
__all__变量   %针对*
import package.time
pip install opop 

第十章
import json
json.dumps(  ,ensure_ascii = False)  % to json字符串 (中文) 
lson.load()  % to python 

gallery.pyecharts.org   %图表网站
from pyecharts.charts import Line
from pyecharts import options
line = Line()
line.add_xaxis(["aaa","bbb","ccc"])
line.add_yaxis("AAA",[12,12,2123] , label_opts = LabelOpts(is_show = False))
line.render()
line.set_global_opts(
    title_opts = TitleOpts(title = "56565" , pos_left = "center" , pos_bottom = "1%"),
    legend_opts = LegendOpts(is_show = True),
    toolbox_opts = ToolboxOPts(is_show = True),
    visualmap_opts = VisualmapOpts(is_show = True)
)
ab173.com

第十一章
from pyecharts.charts import Map
map = Map()
data = [,,,,("上海", 199)]
map.add("地图",data , "china")

from pyecharts.charts import Bar
bar.reveral_axis()

mylist.sort(key = function #lambda element : element[1], reverse = True)

第二阶段   
第一章
pass
class sososo:   %设计类
____name = None
stu = sososo()
stu.name = kokoko
 __init__
__str__    %类转字符串:print的调用
    return "jjjjjjjj"
__lt__
    return self.age < other.age
__le__
    return         <=
__e__
__le__
class sososo(898989):      %继承
mylist : list[int,str,tuple] = [1,"5656",()]   #type: int

from typing import Union %联合类型注解

mysql:
from pymysql import Connection
conn = Connection(
host = "localhost",
port=3306,
user="root",
password="qeryuoQUIWEO2580",
autocommit = Ture
)
print(conn.get_server_info())
conn.close()
cursor = conn.cursor()
conn.select_db("test")
cursor.execute("sql 命令")
result = cursor.fetchall()
conn.commit()


